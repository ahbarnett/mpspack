\section{Dirichlet and Neumann eigenvalue problems}
\label{s:evp}


\bfi % ffffffffffffffffffffffffffffffffffffffffffffffffffffffffffffffffffffff
\hspace{-5ex}\mbox{
\ig{width=3in}{figs/rfnDlow.eps}
\qquad
\ig{width=3in}{figs/rfnNlow.eps}
}
\ca{The lowest few Dirichlet (left) and non-trivial
Neumann (right) eigenmodes for
the smooth nonsymmetric domain eigenvalue problem.}{f:rfnmodes}
\efi

\bfi % ffffffffffffffffffffffffffffffffffffffffffffffffffffffffffffffffffffff
\ig{width=\textwidth}{figs/rfnDk30.eps}
\ca{All Dirichlet eigenmodes with eigenfrequencies lying in $[30,31]$ for
the smooth nonsymmetric domain eigenvalue problem.}{f:rfnmodeshigh}
\efi

\bfi % ffffffffffffffffffffffffffffffffffffffffffffffffffffffffffffffffffffff
\ig{width=\textwidth}{figs/triDlow.eps}
\ca{The lowest Dirichlet eigenmodes of a triangle with $k_j\le 20$,
computed in 7 seconds to 9 digit accuracy via the Fredholm
determinant method.}{f:triDlow}
\efi


We now illustrate the solution of some Laplace eigenvalue problems
of the form
$$
-\Delta u = k^2 u    \qquad \mbox{ in } \Omega
$$
where $\Omega$ is a bounded domain,
with homogeneous boundary conditions either
$$
u = 0 \qquad \mbox{ on } \pO   \qquad \mbox{(Dirichet)}
$$
or
$$
u_n = 0 \qquad \mbox{ on } \pO   \qquad \mbox{(Neumann)} ~.
$$
This has a discrete set of non-trivial solutions $u=\phi_j$ (eigenmodes) and corresponding $k=k_j$ (eigenfrequencies), labeled in nondecreasing
frequency order by $j=1,2,\ldots$.
We start with a smooth domain, which (unlike the trefoil domain
used earlier) is non-symmetric. We apply Dirichlet boundary conditions
and set up an eigenvalue problem (EVP) object.
\begin{verbatim}
s = segment.smoothnonsym(80, 0.3, 0.2, 3);
d = domain(s, 1);        % create an interior domain
s.setbc(-1, 'D');        % Dirichlet BC's applied on inside of segment
p = evp(d);              % sets up eigenvalue problem object
\end{verbatim}
There are three main solver methods coded into \mpspack,
all of which are based on layer potentials (boundary integral operators):
\bi
\item {\tt 'fd'}: the Fredholm determinant method, most efficient and convenient for low-lying
eigenvalues (i.e.\ the lowest few hundred).
\item  {\tt 'ntd'}: the Neumann-to-Dirichlet map method \cite{sca}, most efficient
for high-lying Dirichlet eigenvalues (not available for Neumann case).
\item {\tt 'ms'}: the minimum singular-value method. This is less efficient
(around 10 times slower than {\tt fd}),
but serves as an accurate and reliable reference method (it handles
degeneracies better than {\tt fd}).
\ei
To use the first method a double-layer potential basis is needed
for Dirichlet (single-layer would be used for Neumann).
Then to solve for all eigenvalues lying in $2\le k_j \le 10$ the call
is as follows.
\begin{verbatim}
d.addlayerpot(s, 'd');          % basis set appropriate for BCs
p.solvespectrum([2 10], 'fd');  % Fredholm det method
\end{verbatim}
In 3 seconds, 19 eigenvalues are found and stored in {\tt p.kj}.
Their accuracy may be estimated by {\tt p.ej} which shows
around 13 digits at the low end, deteriorating slightly to 9 digits
at the high end: this is due to the choice of $N=80$ which would need
to be increased to handle higher frequencies.
A simple (but crude and not completely rigorous) check whether any eigenvalues were missed is via Weyl's law:
\begin{verbatim}
p.weylcountcheck(p.kj(1),p.kj,d.perim,d.area);
\end{verbatim}
The resulting graph should oscillate with a range of around 1
without making any permanent
jumps of integer sizes; such jumps up indicate missing (or if down, duplicated)
eigenvalues.
This is most useful for testing large numbers of eigenvalues.

No information about the modes $\phi_j$ was stored.
To also compute modes, and then to plot them, use the following.
\begin{verbatim}
o.modes = 1; p.solvespectrum([2 10], 'fd', o);
p.showmodes;
\end{verbatim}
The solution takes 4 sec, and the plotting 2 sec.
Mode boundary functions $\partial_n\phi_j$ are returned in the array
{\tt p.ndj} from the right singular vector of $(I-2D^\ast)$,
and the double-layer density which generates them in {\tt p.coj}
from the left singular vector.
The plotting is done by default
by evaluation of the single-layer part of the
Green's representation formula for $\phi_j$, using the
normal-derivative data, since this doesn't involve any double-layer evaluation.
The result is the left side of Fig.~\ref{f:rfnmodes}.
When modes are requested, the minimum singular values are also
checked for the eigenfrequencies, giving an independent test of the error.
Errors in modes depends on the separation between eigenvalues,
as in matrix eigenvalue problems, and would be better assessed by a convergence
study in $N$.
The mode data on a grid can be accessed via
{\tt [uj gx gy di js] = p.showmodes;}
The mode is $L^2$-normalized on $\Omega$.


For degenerate eigenvalues you will need to set the option
{\tt o.iter = 0} which switches to the usual SVD for mode checking;
this is slower but scales with $N$ the same way, $O(N^3)$.
This is in development.

The Neumann case (switching to {\tt 'N'} in {\tt setbc} and {\tt 's'} in
{\tt addlayerpot}), and solving in $1\le k_j \le 7.8$
gives the right side of Fig.~\ref{f:rfnmodes}.
As before, the eigevalue estimated accuracies vary from 15 to 10 digits.
Note that the lowest $k$ in this range cannot include zero;
of course the Neumann case does have a trivial eigenpair
$k_0=0$, $\phi_0 \equiv 1$.
Mode evaluation is by default done by evaluating the single-layer potential
given by the left singular vector of $(I+2D)$,
since this doesn't involve any double-layer evaluation.
The mode is approximately $L^2$-normalized on $\Omega$, using the
evaluation grid.


When the domain has symmetries, the {\tt 'fd'} method is not so good,
and does not find modes spanning the eigenspace.
Switching to {\tt 'ms'} or {\tt 'ntd'} is more reliable.


To illustrate computing high-lying Dirichlet eigenvalues as in \cite{sca},
set up the Dirichlet case as above then try the following.
\begin{verbatim}
p.updateN(300);
o.eps = 0.1; o.khat = 'r'; o.fhat = 's'; o.modes = 1;
p.solvespectrum([30 31], 'ntd', o);
p.showmodes;
\end{verbatim}
This takes 4 sec to solve and 4 sec to plot, producing
Fig.~\ref{f:rfnmodeshigh}.
For higher-frequencies, fast multipole evaluation of the modes is
much more efficient. This is not yet documented.

In order to handle corner domain eigenvalue problems, one needs to
use special quadratures,
for instance for a triangle use
\begin{verbatim}
o.kressq = 5;        % corner-packing parameter for Kress reparametrization
s = segment.polyseglist(50, [1, exp(3i*pi/8), exp(5i*pi/4)], 'pc', o);
\end{verbatim}
The result for $k_j \in [4,20]$ is Fig.~\ref{f:triDlow}.
The accuracy of this is currently limited to around 10 digits,
because increasing {\tt kressq}
can result in catastrophic cancellation due to geometry representation
near the corner; this needs to be fixed.
To stop less-accurate modes from being dropped by the solver,
set an option such as
{\tt o.tol = 1e-4}.
A more complicated domain is also illustrated in Section~\ref{s:ridge}.
Neither {\tt 'ntd'} nor {\tt 'ms'} works with corners yet in \mpspack.
See {\tt verg} code on the first author's webpage for the former.

The reader may benefit from also seeing the
codes {\tt test/testevp.m}, {\tt test/testevpms.m}
and the other codes in {\tt examples/} .





\bfi % ffffffffffffffffffffffffffffffffffffffffffffffffffffffffffffffffffffff
\ig{width=\textwidth}{figs/neu_mode_high.eps}
\ca{Neumann eigenmode of a nonsymmetric smooth
planar domain
{\tt s = segment.smoothnonsym(N, 0.3, 0.2, 3);}
with eigenfrequency $k=405.003269518228\ldots$,
around the 42612th eigenvalue of this domain.
}{f:neuhigh}
\efi

\subsection{Method of Particular Solutions and inclusion bounds}

The MPS \cite{mps}
uses basis functions other than layer potentials, and therefore
results in ill-conditioned and/or rectangular matrices.
Ironically, despite the name of this package,
code for the MPS approach to eigenproblems is in a less complete state.

New high-frequency Neumann inclusion bounds
\cite{ninc}
are implemented in the codes in \verb+examples/neumann_inclusion+;
see the {\tt README}. This produces for example, Fig.~\ref{f:neuhigh},
with 15 digits of eigenvalue accuracy.
This is computed by the code \verb+examples/neumann_inclusion/tbl_mfsgsvdincl.m+.
An MFS basis is used, with imaginary displacement for the source point choice.

Work is needed to bring in the various Dirichlet methods into the
{\tt evp.gsvdtension} code which is the cleanest and most accuracte
implementation for the Neumann case.

There are also MPS methods such as {\tt evp.crudempssolvespectrum}
that are incomplete, orphaned, or in development.

We have not tested \mpspack\ for mixed D-N boundary conditions, although
it is probably quite easy to set that up by editing some of the
codes in {\tt @evp}.

%%%%%%

